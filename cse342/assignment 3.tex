
\documentclass[10pt]{article}
\newcommand\Solution[2][\\]{Solution: \Longunderstack[l]{\Question#1#2}}

% TO PROVIDE BLANK SOLUTION
%\renewcommand\Solution[2][\\]{Solution: \setbox0=\hbox{%
%  \Longunderstack[l]{\Question#1#2}}\rule[-\dp0]{0pt}{\dimexpr\ht0+\dp0}}
% TO PROVIDE NO SOLUTION
\renewcommand\Solution[2][\\]{}

\title{342 Assignment 3: NFA}
\begin{document}
\maketitle

Total points: 30\\
Due Date: Feb $20$ 2021\\

The formal definition of a NFA specifies the following:\\
\begin{enumerate}
    \item Alphabet $\Sigma$
    \item the set of states $Q$
    \item the start state $S \in Q$
    \item the accepting states $A \subseteq  Q$
    \item the transition table $\delta$: a function from $Q \times \Sigma$ to $power(Q)$.
    \item reject state: this is usually unspecified, but understood to be included; similar to a return statement at the end of a function.
\end{enumerate}

For every problem in this assignment, 
\begin{enumerate}
    \item $\Sigma = \{a,b,c\}$
    \item $Q = \{S_0,S_1,S_2,S_3,S_4,S_5,S_6,S_7,S_8,S_9\}$
    \item Start state = $S_0$ 
\end{enumerate}
If I am giving the NFA, I will give the accepting states $A$, and the transition table. If you are asked to give the DFA, you need to give $A$ and the transition table.\\ 
For this assignment, give this as a table instead of a diagram. I will give an example in Q1.

NOTE: it is important that you draw and work with diagrams - just translate them when you submit. diagrams give you an insight into  the machine that tables do not.
\newpage
\begin{enumerate}
    \item (10 points) Let $M$ be the NFA with alphabet, set of states and start state as given above. The transition table of $M$ is :
   \begin{center}
      \begin{tabular}{ |c|c|c|c| } 
        \hline
 .  & `a' & `b' & `c' \\\hline \hline
 $S_0$ & $\{S_0, S_1\}$ & $S_0$ & $S_0$\\ \hline
 $S_1$ & - & $S_2$ & - \\ \hline
 $S_2$ & - & - & $S_3$ \\ \hline
 $S_3$ & $S_3$ & $S_3$ & $S_3$ \\ \hline
 $S_4$ & - & - & - \\ \hline
 $S_5$ & - & - & - \\ \hline
 $S_6$ & - & - & - \\ \hline
 $S_7$ & - & - & - \\ \hline
 $S_8$ & - & - & - \\ \hline
 $S_9$ & - & - & - \\ \hline
\end{tabular}
\end{center}
Accepting states $A = \{S_3\}$

For each of the following strings, give the trace (this is the sequence in which each element of the 
\begin{enumerate}
    \item $\epsilon$
    \item  `abc'
    \item `aabcc'
    \item `abcabc'
    \item `aabbcc'
\end{enumerate}

\newpage 
\item (3 points) What is the language of the NFA $M$ from the previous question?, i.e., describe the set of strings accepted by it.
Specifically, 
\begin{center}
L(M) = \{x $\|$ BLANK \}
\end{center}

What is :
\begin{enumerate}
    \item BLANK
\end{enumerate} 


\item (3 points
)(Extra credit) Rename the states to have meaningful names of NFA $M$ from Q1
 \newpage 
 
\item (1+4+2+2+8) Construct a DFA for the  NFA in Q1.
Fill in the ??? in the following. 
When you are making the set of states, you may either give all possible states, and just not use the states that are never visited; or you may start constructing the DFA and list only the states that you are actually using. 
\begin{enumerate}
    \item Alphabet: ??? 
    \item Set of states: ???
    \item Start state: ???
    \item Set of Accepting states: ??? 
    \item transition table: rows are the set of states you gave; columns are the alphabet; entries are from the set of states you gave
\end{enumerate}
\end{enumerate}
\end{document}
