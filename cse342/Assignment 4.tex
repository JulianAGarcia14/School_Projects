

\documentclass[10pt]{article}
\newcommand\Solution[2][\\]{Solution: \Longunderstack[l]{\Question#1#2}}

% TO PROVIDE BLANK SOLUTION
%\renewcommand\Solution[2][\\]{Solution: \setbox0=\hbox{%
%  \Longunderstack[l]{\Question#1#2}}\rule[-\dp0]{0pt}{\dimexpr\ht0+\dp0}}
% TO PROVIDE NO SOLUTION
\renewcommand\Solution[2][\\]{}

\title{342 Assignment 4: e-NFA, Rex}

\begin{document}
\maketitle

Total points: 50\\
Due Date: Mar  $3$ 2021\\

The formal definition of a e-NFA specifies the following:\\
\begin{enumerate}
    \item Alphabet $\Sigma$
    \item the set of states $Q$
    \item the start state $S \in Q$
    \item the accepting states $A \subseteq  Q$
    \item the transition table $\delta$: a function from $Q \times \Sigma \cup \{\epsilon\}$ to $power(Q)$.
    \item reject state: this is usually unspecified, but understood to be included; similar to a return statement at the end of a function.
\end{enumerate}

For every problem in this assignment, 
\begin{enumerate}
    \item $\Sigma = \{a,b\}$
    \itme $Q = \{S_0,S_1,S_2,S_3,S_4,S_5,S_6,S_7,S_8,S_9\}$
    \item Start state = $S_0$ 
\end{enumerate}
If I am giving the NFA, I will give the accepting states $A$, and the transition table. If you are asked to give the DFA, you need to give $A$ and the transition table.\\ 
For this assignment, give this as a table instead of a diagram. I will give an example in Q1.

NOTE: it is important that you draw and work with diagrams - just translate them when you submit. diagrams give you an insight into  the machine that tables do not.
\newpage
\begin{enumerate}
    \item (10 points) Let $M$ be the NFA with alphabet, set of states and start state as given above. The transition table of $M$ is :
   \begin{center}
      \begin{tabular}{ |c|c|c|c| } 
        \hline
 .  & `a' & `b' & `\epsilon' \\\hline \hline
 $S_0$ & - & - & $\{S_1, S_2\}$\\ \hline
 $S_1$ & \{$S_3$\} & - & $\{S_3\}$ \\ \hline
 $S_2$ & - & $\{S_4\}$ & - \\ \hline
 $S_3$ & - & - & $\{S_1, S_5\}$ \\ \hline
 $S_4$ & - & - & $\{S_2, S_5\}$ \\ \hline
 $S_5$ & - & - & - \\ \hline
 $S_6$ & - & - & - \\ \hline
 $S_7$ & - & - & - \\ \hline
 $S_8$ & - & - & - \\ \hline
 $S_9$ & - & - & - \\ \hline
\end{tabular}
\end{center}


Accepting states $A = \{S_5\}$

For each of the following strings, state whether or not the string would be accepted
\begin{enumerate}
    \item $\epsilon$
    \item  `aba'
    \item `aabbb'
    \item `abaaba'
    \item `aabb'
\end{enumerate}

\item (10 points)  For the eNFA in the previous question, give the epsilon closures of the following states
This is the set of states reachable from a given state using only epsilon transitions.(remember, epsilon closure of a state $S$ is a set of state that always includes $S$).
\begin{enumerate}
    \item $S_0$
    \item $S_1$
    \item (1/2 pt) $S_2$
    \item $S_3$
    \item $S_4$
    \item (1/2 pt) $S_5$
\end{enumerate}


\item (10 points) convert the eNFA in the question 1, to an NFA without epsilon transitions.
reminder:  for any pair of states $S$, $T$, there is a transition from $S$ to $T$ on the character "a" iff one the following equivalent conditions hold:
\begin{enumerate}
    \item there is a path from $S$ to $T$ that contains exactly one "a" and all other edges are marked $\epsilon$
    \item there is a path of the form $\epsilon\epsilon\cdot\cdot\cdot\epsilon a\epsilon\epsilon\cdot\cdot\cdot\epsilon$ from $S$ to $T$
    \item for some states $P$ and $Q$, we have $P \in \epsilon-closure(S)$, there is a transition on "a" from $P$ to $Q$ and $T \in \epsilon-closure(Q)$ 
\end{enumerate}
find this out for each character in the alphabet. start state is the old start state. \\

NOTE: accepting state is the set of states whose epsilon-closures contain an accepting state (i.e., any state that can go to an accepting state using only epsilons essentially behaves as an accepting state)

answer:
\begin{itemize}
    \item (1/2 pt) alphabet = ???
    \item (1 pt) set of states = ???
    \item (1/2 pt)start state = ???
    \item (2 pts) accepting states = ???
    \item (6 points) transition table:
       \begin{center}
      \begin{tabular}{ |c|c|c|c| } 
        \hline
 .  & `a' & `b' & `\epsilon' \\\hline \hline
 $S_0$ & - & - & $\{S_1, S_2\}$\\ \hline
 $S_1$ & \{$S_3$\} & - & $\{S_3\}$ \\ \hline
 $S_2$ & - & $\{S_4\}$ & - \\ \hline
 $S_3$ & - & - & $\{S_1, S_5\}$ \\ \hline
 $S_4$ & - & - & $\{S_2, S_5\}$ \\ \hline
 $S_5$ & - & - & - \\ \hline
 $S_6$ & - & - & - \\ \hline
 $S_7$ & - & - & - \\ \hline
 $S_8$ & - & - & - \\ \hline
 $S_9$ & - & - & - \\ \hline
\end{tabular}
\end{center}

\end{itemize} 

\item (10 points) Will the regular expression $(a.a^*|b.b^*)^*$ accept the following?
\begin{enumerate}
    \item $\epsilon$
    \item  `aba'
    \item `aabbb'
    \item `abaaba'
    \item `aabb'
\end{enumerate}

\item (10 points) Write the  regular expression over $\{a,b\}$ for the following languages
\begin{enumerate}
    \item (1 pt) L = \{x|x has the prefix "aab"\}
    \item (1 pt) L = \{x| x has the suffix "aab"\}
    \item (1 pt) L = \{x| x has the substring "aab"\}
    \item (1 pt) L = \{x| x has the subsequence "aab"\}
    \item (2 pt) L = \{x | x either has prefix "a" or suffix "b"\}
    \item (2 pt) L = \{x| x has an odd number of b's \}
    \item (2 pt) L = \{x | length(x) \leq 2\}
\end{enumerate}
\end{enumerate}

\end{document}
 